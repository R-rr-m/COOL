% !TEX program = xelatex
\documentclass[12pt, a4paper]{article}
\usepackage[UTF8]{ctex}
\usepackage{geometry}
\usepackage{fancyhdr}
\usepackage{graphicx}
\usepackage{listings}
\usepackage{xcolor}
\usepackage{booktabs}
\usepackage{tabularx}
\usepackage{hyperref}
\usepackage{amsmath}

% 页面设置
\geometry{left=2.5cm, right=2.5cm, top=2.5cm, bottom=2.5cm}

% 页眉页脚
\pagestyle{fancy}
\fancyhf{}
\fancyhead[L]{COOL 代码生成器实验报告}
\fancyhead[R]{\thepage}
\fancyfoot[C]{\copyright\ 2025}

% 代码高亮设置
\definecolor{codegreen}{rgb}{0,0.6,0}
\definecolor{codegray}{rgb}{0.5,0.5,0.5}
\definecolor{codepurple}{rgb}{0.58,0,0.82}
\definecolor{backcolour}{rgb}{0.95,0.95,0.92}

\lstdefinestyle{mystyle}{
    backgroundcolor=\color{backcolour},   
    commentstyle=\color{codegreen},
    keywordstyle=\color{magenta},
    numberstyle=\tiny\color{codegray},
    stringstyle=\color{codepurple},
    basicstyle=\ttfamily\footnotesize,
    breakatwhitespace=false,         
    breaklines=true,                 
    captionpos=b,                    
    keepspaces=true,                 
    numbers=left,                    
    numbersep=5pt,                  
    showspaces=false,                
    showstringspaces=false,
    showtabs=false,                  
    tabsize=4,
    columns=flexible,
    literate={\_}{{\textunderscore}}1, % 修复下划线问题
    mathescape=false,
    texcl=false
}
\lstset{style=mystyle}

% 标题信息
\title{\textbf{COOL 语言代码生成器开发报告}\\ \large Compiler Principle Assignment PA5}
\author{
    姓名: 谢俊 \\
    学号: 20238131026 \\
    班级: 编译原理课程
}
\date{2025年12月20日}

\begin{document}

\maketitle

\section*{摘要}
本实验在已完成的 COOL 词法分析、语法分析和语义分析的基础上,实现了 PA5 要求的代码生成器(Code Generator)。代码生成器以语义分析阶段输出的带类型标注的抽象语法树(AST)为输入,生成可在 SPIM 模拟器上运行的 MIPS 汇编代码。实验在不修改框架核心文件的前提下,主要扩展并实现了 \texttt{cgen.cc}、\texttt{cgen.h}、\texttt{cool-tree.h}、\texttt{cool-tree.handcode.h} 中的类构建、运行时布局、环境管理以及各类表达式节点的代码生成逻辑。通过与官方编译器 \texttt{coolc} 生成的汇编在 SPIM 上对比测试,验证了实现的正确性与官方实现的一致性。

\section{评分项映射}

\begin{table}[h]
    \centering
    \caption{评分项与实现映射 (代码生成实验)}
    \begin{tabularx}{\textwidth}{l X c}
        \toprule
        \textbf{类别} & \textbf{覆盖点} & \textbf{分值} \\
        \midrule
        代码生成原理 & 目标机模型、调用约定、对象布局、运行时系统接口理解 & 20 \\
        类与对象布局 & 类标签分配、继承树、类名表、对象表、原型对象布局 & 20 \\
        调度表与方法调用 & 分发表构建、静态/动态派发、SELF\_TYPE 处理 & 20 \\
        表达式代码生成 & 赋值、分派、条件/循环、case、算术与逻辑运算等 & 20 \\
        环境与作用域管理 & Environment 设计、本地变量/参数/属性访问、作用域处理 & 10 \\
        测试与集成 & 与官方 coolc 输出对比、SPIM 运行测试、错误定位 & 10 \\
        报告与代码质量 & 结构清晰、逻辑完整、实现与设计一致性 & 10 \\
        主观评分与查重 & 完整性、独立实现程度与可读性 & 10 \\
        \midrule
        \multicolumn{2}{r}{\textbf{总分}} & \textbf{110 (可按课程标准折算为 100)} \\
        \bottomrule
    \end{tabularx}
\end{table}

\section{项目概述与环境}

\subsection{项目目标}
在 COOL 编译器框架下,实现一个与官方编译器 \texttt{coolc} 输出等价的代码生成器,具体目标包括:
\begin{itemize}
    \item 构建完整的类代码生成表(\texttt{CgenClassTable}),分配类标签并建立继承树;
    \item 为每个类生成类名表(\texttt{class\_nameTab})、对象表(\texttt{class\_objTab})、分发表(dispatch table)和原型对象(\texttt{protObj});
    \item 为每个类生成初始化方法(\texttt{Class\_init})和方法实现代码(\texttt{Class.method});
    \item 为各类表达式(赋值、方法调用、条件、循环、case、算术与逻辑运算等)生成符合 COOL 运行时约定的 MIPS 汇编;
    \item 正确处理 \texttt{SELF\_TYPE}、\texttt{isvoid}、\texttt{new} 等特殊语义;
    \item 通过 SPIM 对比测试,验证与官方编译器在行为上的一致性。
\end{itemize}

\subsection{开发与运行环境}
\begin{itemize}
    \item \textbf{操作系统}: Windows 主机 + VMware 虚拟机中的 Ubuntu
    \item \textbf{编译器}: g++ (C++11)、make
    \item \textbf{工具}: 课程提供的 \texttt{lexer}、\texttt{parser}、\texttt{semant}、\texttt{coolc}、\texttt{spim}/\texttt{xspim}
    \item \textbf{工程目录}: \texttt{\textasciitilde/student-dist/assignments/PA5}
\end{itemize}

\subsection{核心文件}
本次实验修改仅以下四个文件:
\begin{itemize}
    \item \texttt{cool-tree.handcode.h}
    \item \texttt{cool-tree.h}
    \item \texttt{cgen.h}
    \item \texttt{cgen.cc}
\end{itemize}

\section{代码生成设计概述}
本节简要说明运行时对象布局、类标签分配、Environment 设计以及表达式代码生成的总体思路,内容与课程 PA5 指南一致。

\subsection{运行时对象布局}
在本次实现中,所有 COOL 对象遵循统一的内存布局:
\begin{itemize}
    \item \textbf{第 0 个字}: 值为 -1 的 eye-catcher;
    \item \textbf{第 1 个字}: 类标签 \texttt{class\_tag},用于 case 匹配与动态类型判断;
    \item \textbf{第 2 个字}: 对象大小(以 word 为单位),包括头部和所有属性;
    \item \textbf{第 3 个字}: 指向分发表 (dispatch table) 的指针;
    \item \textbf{之后若干字}: 按继承顺序排列的属性值。
\end{itemize}
每个类对应一个原型对象 \texttt{Class\_protObj},对象表 \texttt{class\_objTab} 中为每个类保存 \texttt{protObj} 和 \texttt{init} 方法地址,\texttt{new} 表达式通过复制 \texttt{protObj} 并调用对应 \texttt{init} 完成对象创建。

\subsection{寄存器与调用约定}
遵循 COOL 运行时约定:
\begin{itemize}
    \item \texttt{\$a0} (ACC):保存当前表达式求值结果;
    \item \texttt{\$s0} (SELF):保存当前 \texttt{self} 对象;
    \item \texttt{\$fp} (FP):帧指针,指向保存旧 FP 的位置;
    \item \texttt{\$sp} (SP):栈顶指针,栈向低地址增长;
    \item \texttt{\$ra} (RA):返回地址。
\end{itemize}
方法调用时,调用者负责把实参参数压栈,被调用者在栈上为 FP、SELF 和 RA 预留 3 个字的空间,返回时恢复寄存器并弹出参数。

\subsection{垃圾回收 (GC) 策略}
COOL 运行时支持代际垃圾回收(Generational GC)。在代码生成中,我们在以下关键点插入了 GC 支持代码:
\begin{itemize}
    \item 在 \texttt{CgenClassTable} 初始化时,通过 \texttt{MemMgr} 标志选择 GC 策略。
    \item 在对象属性赋值 (\texttt{assign}) 时,如果启用了 GC,调用 \texttt{\_GenGC\_Assign} 宏,通知收集器发生了指针更新(写屏障)。
    \item 确保所有创建对象的请求(\texttt{Object.copy} 或 \texttt{new})都通过运行时系统分配,以便 GC 能够正确追踪。
\end{itemize}

\section{关键实现}
本节从 \texttt{Environment}、\texttt{CgenClassTable}、\texttt{CgenNode} 和典型表达式几个方面,说明具体实现要点,并结合必要的代码片段进行解释。

\subsection{各文件修改内容概览}
\begin{itemize}
    \item \texttt{cool-tree.handcode.h}: 为表达式节点增加类型字段和统一的 \texttt{code} 接口,使语义分析阶段的类型信息可以在代码生成阶段复用;
    \item \texttt{cool-tree.h}: 确认 \texttt{method\_class}、\texttt{attr\_class}、\texttt{branch\_class} 等节点的成员字段(如 \texttt{name}、\texttt{type\_decl}、\texttt{expr})与代码生成访问方式一致;
    \item \texttt{cgen.h}: 新增 \texttt{Environment} 类,并扩展 \texttt{CgenClassTable} 与 \texttt{CgenNode},增加类标签、方法表和属性表等字段;
    \item \texttt{cgen.cc}: 从框架骨架出发,完整实现常量区生成、类表和对象表生成、分发表和原型对象生成,以及所有表达式节点的 \texttt{code} 方法。
\end{itemize}

\subsection{Environment: 环境与作用域管理}
环境类 \texttt{Environment} 在 \texttt{cgen.h} 与 \texttt{cgen.cc} 中实现,负责在代码生成阶段维护局部变量、形式参数和属性的访问信息。

\textbf{关键实现逻辑}:
\begin{itemize}
    \item 使用 \texttt{vars} 向量按栈顺序保存当前作用域内的 \texttt{let} 变量,\texttt{scope\_lengths} 用于记录进入作用域前的栈深度,在退出作用域时恢复。
    \item 使用 \texttt{formals} 保存当前方法的形式参数顺序,便于根据参数索引计算相对于 FP 的偏移。
    \item \texttt{LookUpVar} 从最近作用域向外查找变量在线性表中的位置,换算为相对 SP 的偏移。
\end{itemize}

\begin{lstlisting}[language=C++, caption={LookUpVar 偏移计算逻辑}]
// 查找局部变量偏移 (伪代码)
int Environment::LookUpVar(Symbol name) {
    // 反向遍历 vars 栈
    for (int i = vars.size() - 1; i >= 0; --i) {
        if (vars[i] == name) {
            // 变量在栈顶下方,偏移随压栈深度变化
            return -(vars.size() - 1 - i + 1); 
        }
    }
    return NOT_FOUND;
}
\end{lstlisting}

通过 \texttt{Environment},所有访问变量/参数/属性的表达式都可以使用统一接口,而不必自行计算偏移,降低错误概率。

\subsection{CgenClassTable: 类表与全局代码生成}
\texttt{CgenClassTable} 继承自符号表模板,负责收集所有 \texttt{CgenNode} 并驱动整个代码生成过程:
\begin{itemize}
    \item 在构造函数中调用 \texttt{install\_basic\_classes} 安装 Object, IO, Int, Bool, Str 等基本类,再通过 \texttt{install\_classes} 将用户类包装为 \texttt{CgenNode};
    \item 通过 \texttt{build\_inheritance\_tree} 与 \texttt{set\_relations} 建立父子指针;
    \item 在 \texttt{code()} 中依次调用 \texttt{code\_global\_data}、\texttt{code\_select\_gc}、\texttt{code\_constants}、\texttt{code\_class\_nametab}、\texttt{code\_class\_objTab}、\texttt{code\_dispatchTabs}、\texttt{code\_protObjs}、\texttt{code\_global\_text}、\texttt{code\_class\_inits} 和 \texttt{code\_class\_methods},完成从数据段到文本段的全部输出;
\end{itemize}

\subsection{CgenNode: 类标签、属性与方法布局}
\texttt{CgenNode} 继承自 \texttt{class\_\_class},用于给每个类附加代码生成相关信息。

\textbf{DFS 类标签分配}:
我们采用深度优先搜索(DFS)遍历继承树来分配 Class Tag。这种策略保证了任何类 $C$ 的所有子类的 Tag 都落在区间 $[C.tag, C.max\_child\_tag]$ 内。

\begin{lstlisting}[language=C++, caption={DFS Tag 分配}]
void CgenNode::AssignTags(int& tag_counter) {
    this->class_tag = tag_counter++;
    for (auto child : children) {
        child->AssignTags(tag_counter);
    }
    this->max_child_tag = tag_counter - 1;
}
\end{lstlisting}

这极大地简化了 \texttt{case} 语句和 \texttt{is\_subtype} 的判断逻辑,只需要一次范围检查即可。

\subsection{典型表达式的代码生成}
本实验为所有表达式节点实现了 \texttt{code(ostream\&, Environment)} 方法。下面对若干代表性表达式进行说明。

\subsubsection{静态与动态方法调用 (Dispatch)}
\texttt{dispatch\_class::code} 的实现不仅要处理参数压栈,还要正确处理分发表查找。

\textbf{关键汇编生成逻辑}:
\begin{lstlisting}[language=C++, caption={Dispatch 代码生成片段}]
// 1. 压入参数
for(int i = actual->first(); actual->more(i); i = actual->next(i))
    actual->nth(i)->code(s, env);

// 2. 计算接收者 (Receiver)
expr->code(s, env);

// 3. 检查 void (运行时错误处理)
emit_bne(ACC, ZERO, label_not_void, s);
// ... 调用 _dispatch_abort ...

// 4. 加载 Dispatch Table 并跳转
emit_load(T1, 2, ACC, s);      // T1 = Dispatch Table Ptr
emit_load(T1, method_offset, T1, s); // T1 = Method Address
emit_jalr(T1, s);              // 跳转执行
\end{lstlisting}

\subsubsection{Case 表达式与分支排序}
为了实现 \texttt{case} 表达式中“选择最具体匹配类型”的语义,我们在生成代码前对分支进行了排序。

\textbf{实现策略}:
将所有 \texttt{branch} 按其类型的继承深度(Inheritance Depth)从大到小排序。生成的代码依次检查对象的 Class Tag 是否在分支类型的 Tag 区间内。由于深度大的类(子类)先被检查,一旦匹配成功,必然是最具体的类型。

\begin{lstlisting}[language=C++, caption={Case 分支排序}]
// 使用 lambda 或仿函数进行排序
std::sort(cases.begin(), cases.end(), [&](Branch a, Branch b) {
    return class_table->depth(a->type_decl) > class_table->depth(b->type_decl);
});
\end{lstlisting}

\subsubsection{Let 表达式与栈帧管理}
\texttt{let} 表达式需要在当前栈帧上为新变量分配临时空间。我们的实现并不移动 FP,而是直接调整 SP。
\begin{itemize}
    \item \textbf{进入 Let}:\texttt{env.EnterScope()},生成初始化代码,结果入栈 (\texttt{sw \$a0 0(\$sp); addiu \$sp \$sp -4})。
    \item \textbf{Let Body}:在扩展后的环境中生成 Body 代码。
    \item \textbf{退出 Let}:\texttt{env.ExitScope()},恢复栈指针 (\texttt{addiu \$sp \$sp 4})。
\end{itemize}

\section{遇到的问题与解决方案}

\subsection{段错误 (Segmentation Fault) 与全局初始化}
\textbf{问题描述}:在初步实现 \texttt{CgenClassTable} 后,运行程序立即崩溃。GDB 调试显示崩溃发生在 \texttt{install\_basic\_classes} 内部访问全局指针 \texttt{codegen\_classtable} 时。
\textbf{原因分析}:\texttt{codegen\_classtable} 是一个全局指针,原本应指向当前的 \texttt{CgenClassTable} 实例。但在构造函数执行初期,该指针尚未被赋值。
\textbf{解决方案}:在 \texttt{CgenClassTable} 构造函数的第一行添加 \texttt{codegen\_classtable = this;},确保后续的类安装过程能正确访问全局表。

\subsection{Case 语句匹配逻辑错误}
\textbf{问题描述}:在测试 \texttt{complex\_test.cl} 时,发现多态类型的 \texttt{case} 匹配总是命中父类分支,而不是子类。
\textbf{原因分析}:最初并未对 \texttt{case} 分支进行排序,而是按源码顺序生成。当父类分支出现在子类分支之前时,由于子类也是父类的实例,父类分支会先截获匹配。
\textbf{解决方案}:引入了基于继承深度的排序逻辑,确保“最具体”的类型总是最先被检查。

\section{测试过程与结果}

\subsection{测试策略}
我们采用了分层测试策略:
\begin{enumerate}
    \item \textbf{单元测试}:利用 \texttt{stack.cl} 测试基本的压栈、出栈和算术运算。
    \item \textbf{复杂特性测试}:编写 \texttt{complex\_test.cl} 专门测试递归、继承多态和 Case 语句。
    \item \textbf{回归测试}:使用 \texttt{coolc} 的标准测试套件(如 \texttt{sort_list.cl})确保无退步。
\end{enumerate}

\subsection{命令行测试流程}
在 \texttt{\textasciitilde/student-dist/assignments/PA5} 目录下执行:
\begin{lstlisting}[language=bash]
make clean && make
ln -s ../../bin/cgen ./cgen
./mycoolc -o example_my.s example.cl
../../bin/spim -file example_my.s
\end{lstlisting}

\subsection{终端运行截图}
为直观展示实验过程,本节给出两张实际运行截图。

\begin{figure}[h]
    \centering
    \includegraphics[width=0.8\textwidth]{figures/make_output.png}
    % \fbox{\parbox{0.8\textwidth}{\centering \vspace{2cm} (此处插入 make 与 cgen 编译输出截图) \vspace{2cm}}}
    \caption{在 PA5 目录下执行 make clean 与 make 的编译输出}
    \label{fig:make}
\end{figure}

\begin{figure}[h]
    \centering
    \includegraphics[width=0.8\textwidth]{figures/spim_output.png}
    % \fbox{\parbox{0.8\textwidth}{\centering \vspace{2cm} (此处插入 spim 运行测试结果截图) \vspace{2cm}}}
    \caption{使用 ./mycoolc 编译 example.cl 并在 SPIM 中成功运行的结果}
    \label{fig:spim}
\end{figure}

\section{总结与展望}
本次 PA5 实验实现了 COOL 语言的代码生成器,将语法分析后的抽象语法树成功映射为可在 SPIM 上运行的 MIPS 汇编。报告中详细记录了对 \texttt{cool-tree.handcode.h}、\texttt{cool-tree.h} 和 \texttt{cgen.cc} 的具体修改内容,以及新增的环境管理、类标签分配、分发表构建和各类表达式代码生成功能。

\textbf{主要收获}:
\begin{itemize}
    \item 深入理解了**运行时环境(Runtime Environment)**的设计,特别是栈帧布局和对象模型。
    \item 掌握了**动态分发**的底层实现原理(虚函数表)。
    \item 实践了**汇编级调试**技巧,学会了通过寄存器状态定位逻辑错误。
\end{itemize}

\textbf{未来改进方向}:
目前的实现虽然正确,但在性能上仍有优化空间。例如,可以引入**窥孔优化(Peephole Optimization)**来消除冗余的 \texttt{push/pop} 操作;或者实现更智能的**寄存器分配**策略,减少内存访问次数,进一步提高生成代码的执行效率。

\end{document}
